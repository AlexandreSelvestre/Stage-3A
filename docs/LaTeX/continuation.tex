%%%%
%base : TemplateForSubmission-TECCIENCIA
%Simple template to submit papers in Teccienca
%ROBERT SALAZAR
%%%%%

\documentclass[10pt]{article}

\usepackage[french]{babel}
\usepackage{amsmath}
\usepackage{amsfonts}
\usepackage{amssymb}
\usepackage{mathrsfs}
\usepackage{nccmath}
\usepackage{stmaryrd}
\usepackage{mathtools}
%\usepackage{bbold}
\usepackage{bm}  


\usepackage{fancyhdr}
\pagestyle{fancy}
\fancyhf{}
\rhead{Tecciencia}
\lhead{Template for submission to TECCIENCIA}
\rfoot{Page \thepage}

\usepackage{blindtext}

\usepackage{MnSymbol}

\usepackage[cal=boondox,scr=boondoxo]{mathalfa}

\usepackage{float}
\usepackage{subfig}
\usepackage{fancybox,graphicx}
\usepackage{subfig}
\usepackage{caption}
%\usepackage{subcaption}
\usepackage{color}
\usepackage{authblk}

\usepackage[colorlinks]{hyperref}

%\usepackage{accents}
%\usepackage[titletoc,title]{appendix}
%\usepackage[numbers,sort&compress]{natbib}
%\usepackage{cite}


%floor and ceiling functions
\usepackage{mathtools}
\DeclarePairedDelimiter\ceil{\lceil}{\rceil}
\DeclarePairedDelimiter\floor{\lfloor}{\rfloor}



\usepackage[top=2in, bottom=1.5in, left=1in, right=1in]{geometry}

\begin{document}
%Attention dans la pénalisation lasso à tenir compte de \bm{\beta}_{\text{uni}}

\noindent On constate que l'optimisation de la fonction de perte globale par rapport à $\beta_0$ et $[\mathbf{\beta}^J \; \; \mathbf{\beta}_{\text{uni}}]$ revient à
résoudre
\begin{equation}
\underset{(\beta_0, \beta) \in \mathbb{R} \times \mathbb{R}^{JK + M}}{\text{argmin}} \left( C(\beta_0, (\mathbf{Q}^J)^{-1}\mathbf{\beta},\mathbf{Z}^J \mathbf{Q}^J, \mathbf{y}, \lambda) \right)
\end{equation}
\noindent Où $\mathbf{Q}^J$ et $\mathbf{Z}^J$ sont définis de la façon suivante: 
\begin{align}
\mathbf{Z}^J &= [\mathbf{Z}_{\text{tens}}^J \; \; \mathbf{Z}_{\text{uni}}]\\
\text{with} \; \; \; \mathbf{Z}_{\text{uni}} &= {\mathbf{X}_a}_{:,(JK + 1): (JK + M)}\\
\text{and} \; \; \; \mathbf{Z}_{\text{tens}}^J	&= [\mathbf{Z}_1^J \; \; \hdots \; \; \mathbf{Z}_R^J]\\
\text{where} \; \forall r \in \llbracket 1, R\rrbracket\, , \hspace{0.5 cm} \mathbf{Z_r^J} &= \sum\limits_{k = 1}^K (\underline{X})_{:,:,k} \bm{\beta}_r^K\\
\hspace{7 pt}
\mathbf{Q}^J &= \text{Diag}([\mathbf{u}_{\text{tens}} \; \; \mathbf{u}_{\text{uni}}])\\
\text{with} \; \; \; \mathbf{u}_{\text{uni}} &= (\lambda,\; \; \hdots \; \; , \lambda) \in \mathbb{R}^M\\
\text{and} \; \; \; \mathbf{u}_{\text{tens}} &= (\lVert \mathbf{\beta}_1^K \rVert_1, \; \; \hdots \; \; ,\lVert \mathbf{\beta}_R^K \rVert_1)
\end{align}































\end{document}