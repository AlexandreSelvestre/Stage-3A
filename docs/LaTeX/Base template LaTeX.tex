%%%%
%base : TemplateForSubmission-TECCIENCIA
%Simple template to submit papers in Teccienca
%ROBERT SALAZAR
%%%%%

\documentclass[10pt]{article}
\usepackage[latin1]{inputenc}
\usepackage{amsmath}
\usepackage{amsfonts}
\usepackage{amssymb}
\usepackage{mathrsfs}


\usepackage{fancyhdr}
\pagestyle{fancy}
\fancyhf{}
\rhead{Tecciencia}
\lhead{Template for submission to TECCIENCIA}
\rfoot{Page \thepage}

\usepackage{blindtext}

\usepackage{MnSymbol}

\usepackage[cal=boondox,scr=boondoxo]{mathalfa}

\usepackage{float}
\usepackage{subfig}
\usepackage{fancybox,graphicx}
\usepackage{subfig}
\usepackage{caption}
%\usepackage{subcaption}
\usepackage{color}
\usepackage{authblk}

\usepackage[colorlinks]{hyperref}
\input{doiCmd} %doi command

\usepackage{accents}
\usepackage[titletoc,title]{appendix}
%\usepackage[numbers,sort&compress]{natbib}
\usepackage{cite}


%floor and ceiling functions
\usepackage{mathtools}
\DeclarePairedDelimiter\ceil{\lceil}{\rceil}
\DeclarePairedDelimiter\floor{\lfloor}{\rfloor}



\usepackage[top=2in, bottom=1.5in, left=1in, right=1in]{geometry}
\newcommand{\soft}{\mathcal{S}}
\newcommand{\hard}{\mathcal{H}}
\newcommand*\underdot[1]{ \underaccent{\bullet}{\mathcal{#1}} } %requiere: \usepackage{accents} 
\newcommand*\UnderDot[1]{ \underaccent{\bullet}{#1} } %requiere: \usepackage{accents} 

\usepackage{stackengine}
\newcommand\barbelow[1]{\stackunder[2.5pt]{$#1$}{\rule{1.2ex}{.15ex}}}

\newcommand{\pvec}[1]{\vec{#1}\mkern2mu\vphantom{#1}} %to prime a vector

\newcommand*\UnderTilde[1]{ \underaccent{\sim}{#1} }
  
\renewcommand{\figurename}{Fig.}


%definitions
\def\components{
\left(
\begin{matrix}
    z\cos\gamma_n \vspace{0.3cm} \\
    z\sin\gamma_n\vspace{0.3cm} \\
    -(x-x_n)\cos\gamma_n+(y-y_n)\sin\gamma_n
\end{matrix}
\right)
}

\def\componentsII{
\left(
\begin{matrix}
    z\cos\gamma_n \vspace{0.3cm} \\
    z\sin\gamma_n\vspace{0.3cm} \\
    \mathscr{R}_n\cos(\gamma_n+\beta_n)-x\cos\gamma_n+y\sin\gamma_n
\end{matrix}
\right)
}

\def\change{
\scriptsize
\begin{matrix}
    
    y \rightarrow \Lambda_{n}(\boldsymbol{r})
    \vspace{0.05cm} \\
    r \rightarrow \lambda_{n}(\boldsymbol{r})
\end{matrix}
\normalsize
}



\title{Machine learning methods to help classify liver diseases from radiomics data} 
 
\author[1]{Alexandre SELVESTREL}


\affil[1]{Laboratoire des syst\`emes, Orsay France}
\affil[2]{Centrale-Sup\'elec, Gif-sur-Yvette France}
%\affil[3]{Grupo de F\'isica Te\'orica y desrrollo de Software, Universidad Distrital Francisco Jos\'e de Caldas, Bogot\'a, Colombia}
%\affil[4]{University of Vienna, UniWien, Computational physics group}

\begin{document}

    \date{}
    \maketitle

\begin{abstract}
Rediger l'abstract
\\\\Keywords: Machine learning, Multibloc, Multiway, Liver cancer .  
\end{abstract}


\section{Introduction}
You can use bibtex for references. Single reference is cited as follows \cite{chiaverini2005surface}, and set of references as follows \cite{leibfried2005surface,seidelin2006microfabricated,daniilidis2011fabrication}. Please include the DOI of each reference if it is available.

\blindtext

\section{Mathematical and scientific notation}

\subsection{Displayed equations} Displayed equations should be centered.
Equation numbers should appear at the right-hand margin, in
parentheses:
\begin{equation}
J(\rho) =
 \frac{\gamma^2}{2} \; \sum_{k({\rm even}) = -\infty}^{\infty}
	\frac{(1 + k \tau)}{ \left[ (1 + k \tau)^2 + (\gamma  \rho)^2  \right]^{3/2} }.
	\label{niceEq}
\end{equation}

All equations should be numbered in the order in which they appear
and should be referenced  from within the main text as Eq. (\ref{niceEq}),
Eq. (\ref{LaplaceElectrostaticEquation}). We suggest to use align command for set of equations.

The mathematical problem requires to solve a Laplace's equation for the scalar electric potential $\Phi(\boldsymbol{r})$ given by
\begin{align}
    \nabla^2 \Phi(\boldsymbol{r}) =& 0, &&\boldsymbol{r} \in \mathfrak{D}=\left\{\boldsymbol{r} \in \mathbb{R}^3 : z > 0 \right\}, \label{LaplaceElectrostaticEquation}
\end{align}{}
subjected to the following Dirichlet and Neumann boundary conditions:
\begin{align}
\Phi(\boldsymbol{r}) &= V_o  \hspace{0.5cm} &&\mbox{\textbf{if}} \hspace{0.5cm} \boldsymbol{r} \in \mathcal{A}_{in} \subset \left\{\boldsymbol{r} \in \mathbb{R}^2 : z = 0 \right\},\label{DirichletBoundaryConditionsEq1}\\
\Phi(\boldsymbol{r}) &= 0 \hspace{0.5cm} &&\mbox{\textbf{if}} \hspace{0.5cm} \boldsymbol{r} \in \left\{\boldsymbol{r} \in \mathbb{R}^2 : z = 0 \right\}\setminus\mathcal{A}_{in}\cup\mathcal{G},
\label{DirichletBoundaryConditionsEq2}\\
\frac{\partial \Phi(\boldsymbol{r})}{\partial z} &=0 &&\mbox{\textbf{if}} \hspace{0.5cm} \boldsymbol{r} \in \mathcal{G}.
\label{NeumannBoundaryConditionsEq}
\end{align}

\section{Figures and Tables}
All figures must cited in the manuscript as follows Fig.~\ref{theSystemFig}.


\begin{figure}[H]
\centering %system.pdf
\includegraphics[width=0.45\textwidth]{DarkTeccienciaLogo.png}
    \caption[The system.]{Figure caption. }
\label{theSystemFig}
\end{figure} 

This is a sample of Table.~\ref{analogiesTableEq}. All tables in the document must be cited.

\begin{table}[h]
\begin{minipage}{.95\textwidth}
\begin{center}\small
    \begin{tabular}{ | p{7cm} | p{7cm} | }
    \hline
    \textbf{Electrostatics} & \textbf{Magnetostatics} \\ \hline
    Electric Field & Magnetic field (Biot-Savart law) \\
    $\boldsymbol{E}(\boldsymbol{r}) = \frac{V_o}{2\pi} \mbox{sgn}(z) \int_{\mathscr{G}}  \frac{\boldsymbol{\mathscr{W}}_{\nu}(\boldsymbol{r}')d^2 \boldsymbol{r}' \times (\boldsymbol{r}-\boldsymbol{r}')}{|\boldsymbol{r}-\boldsymbol{r}'|^3}$ & $\boldsymbol{B}(\boldsymbol{r}) = \frac{\mu_o}{4\pi} \int_{\mathscr{G}}  \frac{\boldsymbol{\mathscr{K}}(\boldsymbol{r}')d^2 \boldsymbol{r}' \times (\boldsymbol{r}-\boldsymbol{r}')}{|\boldsymbol{r}-\boldsymbol{r}'|^3}$ \\ \hline
    Weight vector & Surface current density\\
    $\boldsymbol{\mathscr{W}}_{\nu}(\boldsymbol{r})$ 
    & $\boldsymbol{\mathscr{K}}(\boldsymbol{r})$\\ \hline
    Continuity & Continuity (charge conservation) \\
    $\nabla \cdot \boldsymbol{\mathscr{W}}_{\nu}(\boldsymbol{r}) = 0 $
    &$\nabla \cdot \boldsymbol{\mathscr{K}}(\boldsymbol{r}) = 0$ \\\hline    
    Gauss' law & Gauss' law \\
    $\boldsymbol{\nabla}\cdot\boldsymbol{E} = \frac{1}{\epsilon_o}\sigma(\boldsymbol{r})\delta(z)$ 
    & $\boldsymbol{\nabla}\cdot\boldsymbol{B}=0$\\
    \hline
    \end{tabular}
    \caption {The analogy between the gaped SE and magnetostatics.}
    \label{analogiesTableEq}
\end{center}
\end{minipage}%
\end{table}


We use minipage to include several plots in a single figure (see Fig.~\ref{potentialZConstGapedCircularSEFig}).

\begin{figure}[h]  

  \begin{minipage}[b]{0.24\linewidth}
    
    \includegraphics[width=1\textwidth]{DarkTeccienciaLogo.png}
    \caption*{(a) first} %0.5
  
  \end{minipage}
  \begin{minipage}[b]{0.24\linewidth}
    
    \includegraphics[width=1\textwidth]{TeccienciaLogo.png}
    \caption*{(b) second} %vectorFieldPhi05PiN33.pdf
  
  \end{minipage}
  \begin{minipage}[b]{0.24\linewidth}
    \includegraphics[width=1\textwidth]{DarkTeccienciaLogo.png}
    \caption*{(z) third} %vectorFieldPhi3Pi4N33.pdf
    
  \end{minipage}
  \begin{minipage}[b]{0.24\linewidth}
    \includegraphics[width=1\textwidth]{TeccienciaLogo.png}
    \caption*{(d) fourth} %vectorFieldPhi3Pi4N33.pdf
    
  \end{minipage}
  
  
 \caption[Potential.]{Figure caption. }
 
\label{potentialZConstGapedCircularSEFig}
  
\end{figure}


\section*{Conclusions}
\blindtext


\section*{Acknowledgments}
Acknowledgments should be included at the end of the document. The section title should not follow the numbering scheme of the body of the paper. Additional information crediting individuals who contributed to the work being reported, clarifying who received funding from a particular source, or other information that does not fit the criteria for the funding block may also be included; for example, ``K. Flockhart thanks the National Science Foundation for help identifying collaborators for this work.'' 

\bibliographystyle{ieeetr} %alpha, apalike, ieeetr
\bibliography{bibliography.bib}

%\begin{thebibliography}{99} % Bibliography - this is %intentionally simple in this template

%\bibitem{andreotti1997studying} B. Andreotti, \emph{Studying Burgers' models to investigate the physical meaning of the alignments statistically observed in turbulence}, Phys. Fluids \textbf{9} : 3, March (1997)

%\bibitem{cohl1999compact} Cohl, Howard S., and Joel E. Tohline, \emph{A compact cylindrical Green's function expansion for the solution of potential problems}, The astrophysical journal \textbf{527} : 86 - 101 (1999) %DOI: https://doi.org/10.1086/308062

%\bibitem{abramowitz1965handbook} Abramowitz, Milton, and Irene A. Stegun. \emph{Handbook of Mathematical Functions With Formulas, Graphs, and Mathematical Tables.} (1964).


%\end{thebibliography}

%croos section karlie
%GOOD: http://www.eumetrain.org/satmanu/CMs/TrCyAt/print.htm 
%https://physics.stackexchange.com/questions/275799/why-is-the-eye-of-a-cyclone-a-forced-vortex
%http://www.chanthaburi.buu.ac.th/~wirote/met/tropical/textbook_2nd_edition/navmenu.php_tab_9_page_7.1.0.htm
%http://www.atmos.umd.edu/~dalin/andrew/part2.html
%https://nptel.ac.in/courses/119102007/2
%http://www.911omissionreport.com/steering_hurricanes.html
%https://www.youtube.com/watch?v=_brY_9ME8iE brooks
\end{document}