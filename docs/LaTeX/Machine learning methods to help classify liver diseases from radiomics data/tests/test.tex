\documentclass{article}

\begin{document}

Les données sur lesquelles nous travaillons sont issues d'une cohorte de 147 patients atteints de tumeurs du foie. 87 d'entre eux sont atteints d'une tumeur CHC, 23 d'entre eux d'une tumeur CCK et 37 d'entre eux d'une tumeurs mixtes. Chaque patient a réalisé quatre radios du foie, par IRM, une par temps après injection de produit contrastant. Ces quatre temps sont: artériel, portal, veineux et tardif. Cependant, toutes les radios ne sont pas exploitables. En effet, il arrive que le patient bouge durant la radio, ce qui la rend inutilisable. Un tableau récapitulatif est fourni afin de préciser le nombre de radios exploitables par temporalité.

\begin{table}[htbp]
    \centering
    \caption{Nombre de radios exploitables par temporalité}
    \label{tab:example}
    \begin{tabular}{ccccc}
        Patient & Artériel & Portal & Veineux & Tardif \\
        1 & Oui & Oui & Non & Oui \\
        2 & Non & Oui & Oui & Oui \\
        3 & Oui & Non & Oui & Non \\
    \end{tabular}
\end{table}

\subsubsection{Feature extraction in 3D}

\subsubsection{Feature extraction in 2D}

\end{document}